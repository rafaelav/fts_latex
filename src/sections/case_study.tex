\section{Case Study - Concrete examples}


\subsection{Voyager}
\subsection{Cassini-Hygens}
\subsection{Mars Climate Orbiter\cite{mco-nasa}}
Mars Climate Orbiter (MCO) was part of the NASA Mars Surveyor program which
included the Mars Global Surveyor as well as Mars Polar Lander. It was launched
in 1998 with the main objective of studying the climate of Mars. It aimed to
monitor atmospheric dust and water vapor as well as taking pictures of the
planet's surface in order to gain insight on the climatic changes. The hope was
to gather evidences that would suport the theory of existing water underneat the
surface of the planet.

MCO was also designed to serve as a communication relay for the Mars Polar
Lander, however after the Lander's mission, it was supposed to conduct its own
main mission independently.

After its nine months jurney to Mars, MCO was lost beacuse it missed its planned
altitude ordbit and fell into the Martian atmosphere where it was distroyed.

The problem identified in the NASA Mishap Investigation Board Phase I
Report~\cite{mco-rep} was that the MCO failed to use metric units in the coding
of a ground software file used in trajectory models. Other causes that contributed to the failure of the
mission included system engineering, traning and organizational factors.

In order to understand better the circumstances that lead to the failure of this
mission, we need to take a closer look at the Attitude Control System and
the fault protection architecture of the MCO~\cite{surv-nasa-mars}.

The MCO's Attitude Control System contained a sun sensor, an inertial
measurement unit, a star camera and reaction wheels. The MCO was controlable in
roll, pitch and yaw so that it can adjust its alttitude, to perform trajectory
maneuvres and to achieve Mars Orbit Inseration. It was designed so that it could
use the gravity force of Mars in order to adjust to the needed velocity which
could allow it to keep the desired orbit and altitude. The design included
rocket engine modules which allowed this as well as serving as elements that
dissipaed any angular momentum which could be accumulated in the reaction
wheels. The Spacecraft Performance Analysis Software (SPAS) was used to
calculate the angular momentum desaturation, however it was not using the
correct units of measures (error in the N to lb-f conversion). This has lead to
the MCO estimating badly its alttitude and orbit.

The fault protection capabilities were unsatisfying at a number of levels in
this mission. First of all, there was an error in the human managemnt process.
The calculation of the automatic momentum desaturation involved a human
constant, because people where involved in loading the information to servers,
also in retrieving it from servers and uploinking it to the MCO. The teams
dealing with the information where not specifically the ones who introduced the
error, however they could have observed the problem and find a solution to
solved it. This proves the existance of a fault in the human-system interaction.
This can be avoided by developing a better system that would allow people to
asses the correctens of the output data based on the input they are giving. It
could be a software solution that would act either as verifier (just calculate
and report the expected results) or as a checker that would identify the
inconcistency and report it.

The critical problem in the MCO case was the lack of an adequate convertor
between the Newtons and lb-f. The software used for the MCO was partly carried
over from another mission (THe Mars Global Surveyor). The privios mission
considered the convertor, however it was not imported for the MCO mission by mistake. The fault could have been
avoided if there would have been thorough testing after the importing of the
already existing software. If the re-used functions would have been inspected
with attention and tests would have been developed in order to check the correct
functionality in the new context, the fault could have been discovered and
solved.

The expectation to have a software system with no bugs or problems is quite
unrealistic. However this should not affect the success of missions. The MCO was
already calculating the angular momentum desaturations and would afterwards
downlinked it to the ground base. At that point the Spacecraft Performance
Analysis Softer was used to calculate the angular momentum desaturation value
without making the correct conversion between Newtons and lb-f. Howver, the
result would be provided to the team in charge and only afterwards the plan for
the future actions would be uplinked to the spacecraft. In this case, another
solution for fault avoidance would have been the limitation or elimination of
the need of the spacecraft to relay on external systems and the human
intervention. Basically, if the the MCO would be designed as to be able to
calculate its own position and only then cross check it with the avilable data
provided by the ground team, it would have been possible to at least notice and
report the problem.

In order to have a good failure tolerance, it was decided that for future
spacecrafts it would be advised to use redundant and independent components
(both software and hardware). In the case of performing the same calculations
with the use of various software modules it could be possible to identify the
differences in the values. In the case of multiple such components, a voting
could be implemented in order to rule out the modules that provide wrong
results.

Since this case is mainly a failure caused by the inadequate re-use of
software, it can be stated that such errors can be avoided through the use of
independent evaluation and validation of the re-used software. The use of
regression testing~\footnote{http://en.wikipedia.org/wiki/Regression\_testing}
as well as comparing results to expected outputs can prove helpful.
\subsection{Mars Reconnaissance Orbiter\cite{mro-nasa}}

The Mars Reconnaissance Orbiter (MRO) was launched on the 12th of August, 2005
and it was one of the first spacecrafts entering Mars' orbit.

Its journey to Mars lasted seven months and it took six more months afterwards
in order for it to reach its science orbit. The MRO was created with the purpose
of tracking changes of the water and dust in Mars' atmosphere. It was also
supposed to look for more evidence of ancient seas and hot springs and to study
the climate changes based on Mars' surface minerals. It also serves as a data
relay station for other missions.

The objectives of the MRO where complex and the probability of failure was high
because of this. There were multiple things that could go wrong with the
mission, considering the complexity of its overall goals and functionality
expectations \cite{mro-mission}.

First of all, the mission was supposed to last about 5.4 years. The system on the spacecraft was supposed
to have an up-time as close as possible to 100\% during this mission. This is
hard to achieve because all possible failures need to be prevented or handled
without losing functionality or precious data.

An analysis of the fault pretection architecture, the effectiveness of this
architecture as well as a categorization of the fault protection capabilities of
the MRO are described in detail in \cite{surv-nasa-mars}.

In order to assure that no big failures cand occur and that, despite its
complexity, the system will work accordingly in order to achieve the mission's
goals, a semi-autonomuos fault protection software (SPIDER - SPacecraft Imbedded
Distributed Error Response) has been developed for the MRO. It has been
developed in C-language with the possibility of re-use for any other future
missions. SPIDER can support redundancy and cross strapping tasks, requirements
which the MRO is supposed to meet.

Despite it's advanced capabilities and the fact that it acts as a first
responder to erros and handles most of them, SPIDER was not design to function
on its own. It still needed help from the ground in order for the system to be
set back in a normal state for the cases when it entered a safe mode. Also,
ground operations where needed in case the system did not recognise some
possibly threatening conditions that could appear. The main advantage of the
SPIDER is that it can actually prevent the system from entering massive failure
without the need to wait for the ground crew to notice and react to the
problems.

FIG - Generic SPIDER Decission process \cite{e-seale}

The SPIDER ensures that the fault responses are general. For example, if the MRO
finds itself in a position in which a fault is detected, than it automatically
switch to a safe mode and interrupt all the uncritical equipment and
functionality. The system can be turned back to normal functionality from the
ground level by the team in charge after following a defined protocol.

The SPIDER ensures that the MRO is not affected by fails at any given time, with
the presumption that only one system can generate failure at a given time. It
basically tries to ensure that the MRO is completely single-fault tolerant and
this leads to the robustness and high relability level of the architecture. As
it was mentioned before, SPIDER was created around the idea of keeping things
redundant and around the cross strapping concept. Among the redundancy technics
it uses we can find:
\begin{itemize}
\item Block Redundancy - which ensures that a failure can be solved through the
use of parallel elements
\item Functional Redundancy - which permits the handling of a failure in various
ways
\item Cooperative Redundancy - which permits the division of a system function
in more elements. This helps by allowing the function to be able to succeed despite
the probability of having one of its elements fail
\end{itemize}

The architechture of the SPIDER is hierarchicle. It has three software levels:
the component level fault protection which is used for communication with the
MRO hardware, the performance level fault protection which keeps track of the
performance of each subsystem and the system level fault protection which tries
to keep failures from happening. Depending on the type of failure, it can be
handled by the befitting logic. In order to avoid starvation, the higher levels
in the hierarchy are allowed to call on to the lower levels in order to assign
them specific tasks, but they cannot take priority of the tasks which
already are executing in these levels.

The SPIDER has been thorughly tested before being integrated on the MRO, but a
final proof of its capability was the fact that, even though the MRO has
encountered quite a few failures during the mission (e.g. memory corruption,
downlink connectivity failure, etc.), it was capable to recover and to achieve
the mission's goals.

\subsection{Mars Exploration Rovers\cite{mer-nasa}}

NASA has launched in 2003 the twin rovers, Opportunity and Spirit, with the
purpose of gaining more insight about past water activity on the red planet and
if there where, at some point, conditions that could favorite life on our
neighbor planet. The planned mission was supposed to last for about 3 months
from the moment the rovers would land. However, they have exceeded any
expectations both as far as their durability is concerned as well as the
discoveries they have made.

Spirit has collected evidences that, in the past, Mars was much wetter than we
see it know. Also, it collected information about the wind on the red planet.
The rover has been silent as of March 2010 and the ground team has ceased to try
to contact it since May 2011 when its mission was closed and considered
complete.

Opportunity has found evidence that Mars could have been capable of sustaining
microbial life. During its mission it traveled for over 20 km (as of March 2010)
in its search for a better understanding of the planet. The rover is still
active (as of 5th November 2013).

An overview of why the fault protection worked so well for these rovers is
provided in \cite{surv-nasa-mars}.

The rovers needed to enter a into a sleep mode days in order for them to be able
to recharge their batteries. In this state their CPU would be powered off, but
the hardware still needed to maintain the safe thermal and power states (the
temperatures on Mars can be quite extreme - from negative
107$\,^{\circ}\mathrm{C}$ up to over positive
30$\,^{\circ}\mathrm{C}$\footnote{http://en.wikipedia.org/wiki/Climate\_of\_Mars}).

Upon waking from the sleep mode, the rovers were expected to start the
communication without any help from ground operation teams on Earth. As a fault
protection measure, if during the rover's initialization a server error occured,
the reset of the system would be postponed for a pre-defined amount of time. A
software health function was also use in order to check for any lingering or
suspended tasks. For the communication system, redundancy was used such that, in
case the high gain antenna would not function properly then the low gain antenna
would perform its tasks (even if this one had lower data rates it could still be
enough so that the mission wouldn't be a failure).

The local and system level fault protection can be seen in Fig. \cite{fprot}

FIG. Overview of the surface fault protection

There are more general capabilities of the fault protection architecture used in
the case of the rovers. First of all, the ground teams can monitor the data
which the rovers send back (or fail to send back) and can intervene based on the
type of errors they observe.

Since the rovers have an algorithm that controls autumatically the wake up and
shutdown procedures, it means that a low energy fault can be avoided. The
algorithm only allowed the rovers to work when they haved enough energy to do
this, thus abnormal terminations or uncapability to save data before entering
sleep mode can were avoided. 

A plus to the fact that the rovers needed to enter sleep mode everyday and
basically shutdown for a period of time was software
rejuvenation\footnote{http://en.wikipedia.org/wiki/Software\_aging}. Memory
leaks could be removed through this process.

The rovers were also tolerant to failure because, for example, in case they did
not stock up enough energy in order to function properly, than the system would
not allow the to wake up from the sleep mode and by doing this it avoided the
failure of letting them work without appropriate power. Aside from this, they
benefited from the presence of a navigation algorithm that looked for possible
hazards on the rovers' paths and tried to avoid them by chosing between posible
identified routes.

Even thogh testign was developed previous to mission launch, complete fault
avoidance could not be guaranteed. The ground team could identify that after 17
sols (Martian days) one of the rovers was rebooting over and over. They managed
to identify the problems as being the fact that when a file needed to be delited
from RAM, even though the link to it was lost, the sizse of the table of
contents of the RAM was not decreased so the system managed to consume all the
RAM space available after some time. The problem was solved from teh ground when
the team instructed to software to not use the memory. This prevented the
failure of the mission.
