\section{Case Study - Concrete examples}


\subsection{Voyager}
\subsection{Cassini-Hygens}
\subsection{Mars Climate Orbiter\cite{mco-nasa}}
Mars Climate Orbiter (MCO) was part of the NASA Mars Surveyor program which
included the Mars Global Surveyor as well as Mars Polar Lander. It was launched
in 1998 with the main objective of studying the climate of Mars. It aimed to
monitor atmospheric dust and water vapor as well as taking pictures of the
planet's surface in order to gain insight on the climatic changes. The hope was
to gather evidences that would suport the theory of existing water underneat the
surface of the planet.

MCO was also designed to serve as a communication relay for the Mars Polar
Lander, however after the Lander's mission, it was supposed to conduct its own
main mission independently.

After its nine months jurney to Mars, MCO was lost beacuse it missed its planned
altitude ordbit and fell into the Martian atmosphere where it was distroyed.

The problem identified in the NASA Mishap Investigation Board Phase I
Report~\cite{mco-rep} was that the MCO failed to use metric units in the coding
of a ground software file used in trajectory models. Other causes that contributed to the failure of the
mission included system engineering, traning and organizational factors.

In order to understand better the circumstances that lead to the failure of this
mission, we need to take a closer look at the Attitude Control System and
the fault protection architecture of the MCO~\cite{surv-nasa-mars}.

The MCO's Attitude Control System contained a sun sensor, an inertial
measurement unit, a star camera and reaction wheels. The MCO was controlable in
roll, pitch and yaw so that it can adjust its alttitude, to perform trajectory
maneuvres and to achieve Mars Orbit Inseration. It was designed so that it could
use the gravity force of Mars in order to adjust to the needed velocity which
could allow it to keep the desired orbit and altitude. The design included
rocket engine modules which allowed this as well as serving as elements that
dissipaed any angular momentum which could be accumulated in the reaction
wheels. The Spacecraft Performance Analysis Software (SPAS) was used to
calculate the angular momentum desaturation, however it was not using the
correct units of measures (error in the N to lb-f conversion). This has lead to
the MCO estimating badly its alttitude and orbit.

The fault protection capabilities were unsatisfying at a number of levels in
this mission. First of all, there was an error in the human managemnt process.
The calculation of the automatic momentum desaturation involved a human
constant, because people where involved in loading the information to servers,
also in retrieving it from servers and uploinking it to the MCO. The teams
dealing with the information where not specifically the ones who introduced the
error, however they could have observed the problem and find a solution to
solved it. This proves the existance of a fault in the human-system interaction.
This can be avoided by developing a better system that would allow people to
asses the correctens of the output data based on the input they are giving. It
could be a software solution that would act either as verifier (just calculate
and report the expected results) or as a checker that would identify the
inconcistency and report it.

The critical problem in the MCO case was the lack of an adequate convertor
between the Newtons and lb-f. The software used for the MCO was partly carried
over from another mission (THe Mars Global Surveyor). The privios mission
considered the convertor, however it was not imported for the MCO mission by mistake. The fault could have been
avoided if there would have been thorough testing after the importing of the
already existing software. If the re-used functions would have been inspected
with attention and tests would have been developed in order to check the correct
functionality in the new context, the fault could have been discovered and
solved.

The expectation to have a software system with no bugs or problems is quite
unrealistic. However this should not affect the success of missions. The MCO was
already calculating the angular momentum desaturations and would afterwards
downlinked it to the ground base. At that point the Spacecraft Performance
Analysis Softer was used to calculate the angular momentum desaturation value
without making the correct conversion between Newtons and lb-f. Howver, the
result would be provided to the team in charge and only afterwards the plan for
the future actions would be uplinked to the spacecraft. In this case, another
solution for fault avoidance would have been the limitation or elimination of
the need of the spacecraft to relay on external systems and the human
intervention. Basically, if the the MCO would be designed as to be able to
calculate its own position and only then cross check it with the avilable data
provided by the ground team, it would have been possible to at least notice and
report the problem.

In order to have a good failure tolerance, it was decided that for future
spacecrafts it would be advised to use redundant and independent components
(both software and hardware). In the case of performing the same calculations
with the use of various software modules it could be possible to identify the
differences in the values. In the case of multiple such components, a voting
could be implemented in order to rule out the modules that provide wrong
results.

Since this case is mainly a failure caused by the inadequate re-use of
software, it can be stated that such errors can be avoided through the use of
independent evaluation and validation of the re-used software. The use of
regression testing~\footnote{http://en.wikipedia.org/wiki/Regression\_testing}
as well as comparing results to expected outputs can prove helpful.
\subsection{Mars Reconnaissance Orbiter\cite{mro-nasa}}

The Mars Reconnaissance Orbiter (MRO) was launched on the 12th of August, 2005
and it was one of the first spacecrafts entering Mars' orbit.

Its journey to Mars lasted seven months and it took six more months afterwards
in order for it to reach its science orbit. The MRO was created with the purpose
of tracking changes of the water and dust in Mars' atmosphere. It was also
supposed to look for more evidence of ancient seas and hot springs and to study
the climate changes based on Mars' surface minerals. It also serves as a data
relay station for other missions.

The objectives of the MRO where complex and the probability of failure was high
because of this. There were multiple things that could go wrong with the
mission, considering the complexity of its overall goals and functionality
expectations \cite{mro-mission}.

First of all, the mission was supposed to last about 5.4 years. The system on the spacecraft was supposed
to have an up-time as close as possible to 100\% during this mission. This is
hard to achieve because all possible failures need to be prevented or handled
without losing functionality or precious data.

An analysis of the fault pretection architecture, the effectiveness of this
architecture as well as a categorization of the fault protection capabilities of
the MRO are described in detail in \cite{surv-nasa-mars}.

In order to assure that no big failures cand occur and that, despite its
complexity, the system will work accordingly in order to achieve the mission's
goals, a semi-autonomuos fault protection software (SPIDER - SPacecraft Imbedded
Distributed Error Response) has been developed for the MRO. It has been
developed in C-language with the possibility of re-use for any other future
missions. SPIDER can support redundancy and cross strapping tasks, requirements
which the MRO is supposed to meet.

Despite it's advanced capabilities and the fact that it acts as a first
responder to erros and handles most of them, SPIDER was not design to function
on its own. It still needed help from the ground in order for the system to be
set back in a normal state for the cases when it entered a safe mode. Also,
ground operations where needed in case the system did not recognise some
possibly threatening conditions that could appear. The main advantage of the
SPIDER is that it can actually prevent the system from entering massive failure
without the need to wait for the ground crew to notice and react to the
problems.

FIG - Generic SPIDER Decission process \cite{e-seale}



\subsection{Mars Exploration Rovers}