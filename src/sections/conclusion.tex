\section{Conclusions}
This paper aimed to take a closer look into the fault-tolerance techniques used
in space missions. The need to explore our Universe as far as we can reach will
persist as long as we humans will exist. This means that more and more space
missions will be conducted and, in order for them to be successful, we need to
keep in mind what went good in previous missions, what we should use again, but
also what were our mistakes and how to avoid them in the future.

We have presented some of the conditions which make space missions special as
well as some general architectures and fault-tolerance techniques which can be
applied to most of them. But, since missions are in general very different
because of multiple reasons like goals, duration, target distance etc., they
mostly need to face different or even unknown problems and as such, solutions
need to be always adapted for each mission. We have, therefore taken a closer
look at some specific mission and what they have used as far as fault tolerancy
goes.

We have seen that even the smallest mistake can lead to a mission failure, but
also that even though software will always contain small buggs, a good and
healthy fault tolerant design can ensure the success of a mission, despite some
problems which might occure from time to time.

As a last thought, when it comes to space missions, every detail, no matter how
small, needs to be taken into consideration, all theories need to be tested
thoroughly and every possibility need to be taken into account. However, the
work this involves will always be richly rewarded each time we will find a new
piece in the puzzle that our Universe represents.
