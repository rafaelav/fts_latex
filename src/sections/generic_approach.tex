\section{General approach to fault protection in spacecrafts}
The characteristics of space exploration missions have always required
innovative fault management (FM) strategies\footnote{The term \textit{Fault
Management} is often used interchangeably with the term \textit{Fault
Protection} in NASA documentation.} in order to achieve mission success.
The used FM strategies and implementations vary significantly from mission to
mission, but there are a set of general characteristics that influence the
design of the FM. Engineers from NASA have been working on a Fault Management
Handbook with the purpose of being \textit{`a guidance document to provide
guidelines and recommendations for defining, developing, analyzing, evaluating,
testing, and operating the Fault Management (FM) element of flight
systems`}\cite{nasa-fm-handbook}.
 
According to \cite{nasa-fm-handbook}, a space missions FM engineer has the role
of analysing mission attributes and characteristics and defining FM
requirements, as well as developing hardware and software architectures that
provide fault protection and assure the success of missions. This process is
illustrated in Figure~\ref{fig:fm_design}.

\begin{figure}[htb] 
	\begin{center}
	\includegraphics[width=0.8\textwidth]{img/fm_design.png}
	\caption{Fault Management design process. \small{\textit{Source: NASA Fault
	Tolerance Handbook, draft 2, 2012}}}
	\label{fig:fm_design}
	\end{center}
\end{figure}

The first step during development of the FM for a space mission is a clear
identification of mission goals and of functions and resources that are critical
and need to be protected\footnote{These include functions, resources, events
which are needed to achieve the mission's goals. E.g.: successful launch,
scientific payload, collected data etc. }. Subsequently, the FM engineers need
to perform an analysis of the system architecture to identify the mission
characteristics which are definitory for the fault protection functions. Fault
management will need to be implemented throughout the system, however they need
to take into account restrictions enforced by the mission characteristics. For
example, in the case of deep space exploration missions, which have delayed and
infrequent contact with the mission command on Earth, or during time critical
steps of missions (e.g. planetary atmospheric entry), the \textbf{response
latency} of fault management is critical. According to the NASA Fault Management
Handbook \cite{nasa-fm-handbook}, response latency is \textit{'the time from the
occurrence of a fault to the correction of the failure condition'} and is
\textit{'built from the various mission and system characteristics that impact
the execution of each of the core FM functions'}. To assure system reliability,
the FM response needs to either clear or contain the effects of a failure before
the failure is propagated to other components or before system issues occur.

Over the years, the technologies used for fault protection have been
significantly improved, but the fundamental principals of spacecraft design have
suffered little changes. For example, the fault tolerance design at the NASA Jet
Propulsion Laboratory is based on the following \textbf{fault management
principles} (quoted from \cite{fm-jpl}):
\begin{enumerate}
  \item Respond only to unacceptable conditions
  \item Avoid hair triggers and re-triggering
  \item Tolerate false alarms
  \item Make parameters commandable
  \item Corroborate before severe responses
  \item Ensure commandability and long term safety
  \item Preserve consumables and critical data
  \item Log events and actions
\end{enumerate}

The next step followed by the FM engineers is to design and implement software
and hardware solutions to assure the fault protection of a spacecraft. A series
of techniques can be used in different situations and at different moments of
time (different stages). In the presentation of Fault Management at the Jet
Propulsion Lab\cite{fm-jpl}, John Day and Michel Ingham detail a hierarchy of
approaches and give examples of a series of strategies that may be used to
assure fault protection. The hierarchy can be seen in
Figure~\ref{fig:fault_protection_hierarchy}. A first strategy that should be
followed is to to avoid faults altogether, through \textbf{fault avoidance}
techniques. However, for cases when faults do occur, \textbf{fault tolerance}
needs to be in place, to assure that any faults have minimal effects on the ship
and do not cause a mission failure. From this category, for example,
\textbf{Fault Masking}, \textbf{Gradual Degradation} and \textbf{Fault
Detection, Isolation and Recovery} (FDIR) are some general strategies that can
be employed.

\begin{figure}[htb]
	\begin{center}
	\includegraphics[width=0.8\textwidth]{img/fault_protection_hierarchy.png}
	\caption{Fault protection general hierarchy. \small{\textit{Source: J.
	Day and M. Ingham, Fault management at JPL: past, present and future, 2011}}}
	\label{fig:fault_protection_hierarchy}
	\end{center}
\end{figure}

The design process is one where, mainly because of costs and time restrictions,
usually engineers start from already developed and previously used techniques
and technologies and improve them, increasing their safety, reliability and
accuracy. For example, the most important strategy used for FM is
\textbf{Redundancy}, since the first space missions. This is usually implemented
in multiple ways\cite{surv-nasa-mars}:
\begin{itemize}
\item Block Redundancy - which ensures that a failure can be solved through the
use of parallel elements
\item Functional Redundancy - which permits the handling of a failure in various
ways
\item Cooperative Redundancy - which permits the division of a system function
in more elements. This helps by allowing the function to be able to succeed despite
the probability of having one of its elements fail
\item Cross Strapping - which permits the system to handle multiple failures
that occur in the system.
\end{itemize}

In the following chapter, we will go into the details of some of the FM
implementations for former NASA deep space exploration missions. However, before
that, a look at the typical execution architecture of spacecrafts is beneficial.
Figure~\ref{fig:spacecraft_execution_architecture} depicts such a generic
architecture, as seen in \cite{fm-jpl}. As it can be observed, command
sequences\footnote{Low-level commands or macros defining actions that need to be
executed for the mission completion.} are being executed by a \textit{nominal
sequencing engine}. Simultaneously, fault protection software is run and is
always prepared to take over the control and execute fault protection actions,
if the behaviours indicate the presence of a fault. Typically, the fault
protection is handled by the Command and Control Subsystem.

\begin{figure}[ht]
	\begin{center}
	\includegraphics[width=0.6\textwidth]{img/spacecraft_execution_architecture.pdf}
	\caption{Typical Spacecraft Execution Architecture. \small{\textit{Source: J.
	Day and M. Ingham, Fault management at JPL: past, present and future, 2011}}}
	\label{fig:spacecraft_execution_architecture}
	\end{center}
\end{figure}

The design and implementation of fault management needs to be followed by
Verification and Validation, to prove that the system complies with the system
requirements and that the response to failures is the intended one. Besides the
mission specific requirements, all space missions have to respect some general
safety requirements. Mandatory ones include the \textbf{NASA General Safety
Program Requirements} (NPR-8715.3) and the \textbf{Eastern and Western Range
Safety Requirements} (EWR 127.1)\footnote{EWR-127.1 represents the specification
of safety requirements for the launch site. It is being replaced with the
\textbf{Air Force Space Command Manual Range Safety User Requirements}
(AFSCPCMAN 91-710).}\cite{surv-nasa-mars}. In addition, a multitude of other
documents include recommendations for requirements related to fault protection
in the case of spacecrafts. Some examples include: Rules for the Design,
Development, Verification, and Operation of Flight Systems (GSFC-STD-1000E),
NASA Systems Engineering Processes and Requirements (NPR 7123.1A), Risk
Classification for NASA Payloads (NPR 8705.4) etc.
