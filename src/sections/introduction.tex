\section{Introduction}
For more than 50 years the human kind has been sending ships into the outer
space in order to look for answers to centuries-old questions about what is
beyond our atmosphere. And we have managed to explore and understand more and
more of our Solar System and the space beyond. The successful missions have been
the result of the reliable architectural design, implementation and testing done
by the engineers working on them.

The extremely high costs, in both money and time, and the high risks of such
missions require an extremely high level of reliability that can guarantee a
very high chance of success. Thus, fault protection of the systems involved in
these missions is of utter importance.

Spacecrafts are extremely complex and their design is usually very specific to
the particular mission objective. Furthermore, space missions are meant to have
a duration of months or years and sustain operation in extreme environment and
under harsh conditions. This, in addition to financial and temporal limitations,
make the design of fault protection mechanisms a challenging task for the
engineers. When taking into consideration the deep space exploration missions
(i.e missions meant to last for multiple years and have the purpose of exploring
the Solar System and beyond), an even higher level of reliability is required
and various techniques to assure fault prevention, detection, containment and
recovery need to be employed to assure mission success.

This paper aims to take a look into the fault protection approaches used in the
spacecrafts domain, with a particular focus on deep space explorations missions.
A general look at the characteristics of such missions is first taken, followed
by a brief introduction to the general approaches followed when designing and
implementing fault protection for spacecrafts. Then, a series of previous space
missions (both successful and unsuccessful) are analyzed, with a focus on the
fault management.
