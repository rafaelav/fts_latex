\section{Requirements for spacecrafts}
The engineers working on the construction of spacecrafts have to face a
multitude of requirements which are specific to this domain and cannot be found
in other, more 'traditional' fields. Thus, in order to properly understand the
specifics of fault-tolerant approaches used for space-related applications, this
chapter will briefly describe these requirements.

One of the most important characteristics of this domain is that it deals with a
lot of uncertainty. While we have a strong understanding of the Earth
environment (and even surrounding space), moving farther away from our home
planet brings any ship in an environment where unexpected factors can influence
a mission. Thus, the ships must be prepared to respond to conditions for which
it might not have been initialy designed.

Another important factor is the harshness of the environment and the extreme
conditions present in space which may have dangerous effects. A few examples
include\cite{req-space-environment}:
\begin{itemize}
  \item extremely low or extremely high temperatures (usually due to incoming
  solar radiation, reflected solar energy or friction when entering a planetary
  atmosphere) can affect various instruments or compontents
  \item ionized gas or plasma can charge the surface of a spaceship and
  disrupt the operation of electrical instruments
  \item meteoroids and orbital debris may cause damage and affect the structural
  integrity of ships if hit\footnote{The speed with which a meteoroid or debris
  are hit can also majorly influence the effects.}
  \item ionizing radiation consists of high-energy particles that can `travel
  through spacecraft material and deposit kinetic energy`.
  \item harmful/toxic substances in the atmosphere of other planets may also
  cause corosion and affect the proper function of components
\end{itemize}

Spacecrafts which are travelling far away from Earth also face other issues due
to operating with limited ground contact (mostly because of large distances to
our planet). According to \cite{fm-jpl}, these include:
\begin{itemize}
  \item Extended periods with no planned contact (1 to 4 weeks)
  \item Planned contact periods may be short (1 to 2 hours)
  \item Ground may not show for planned contacts (5\% to 10\%)
  \item Large one-way light times (minutes to hours)
  \item Low downlink data rates (10 to 40 bps)
\end{itemize}
For example, light travels 2h and 50 minutes to reach Saturn from Earth so
autonomous operation was required for probes such as Voyager when at such
distances, as no 'live' commands could be received from earth if any situation
had occured.

Two other aspects that are characteristic to long-term space missions is that
there are no humans onboard them (unmaned) and they need to survive without any
maintenance during the entire mission time, which can last for more than a
couple of years\footnote{For example, both Voyager 1 and Voyager 2 have been
active for more than 36 years.}. No spare parts are available and no humans can
directly intervene to fix any issues that might appear, so the ships have to
deal with this using other methods (e.g. redundance).

Lastly, we can add a couple of other characteristics of spacecrafts: 
\begin{itemize}
  \item power supply is usually limited and the generated energy needs to be
  divided to all the instruments and components of the ship
  \item need to keep costs down as all components are extremely expensive to produce
  \item space usually needs to be restricted and its use optimized
\end{itemize}

As we have seen, there are a multitude of characteristics of space missions (and
long-term space missions in particular) and a series of specific requirements,
all of which need to be taken into account when designing spacecrafts. Thus, all
of these have direct influence on the fault tolerance mechanisms used.
